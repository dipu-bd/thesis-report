\documentclass{standalone}
\usepackage{standalone}

\begin{document}
\chapter{Introduction}
In the field of Image processing and computer vision license plate recognition is an well established title. This thesis tries to implement this established title for license plates written in Bengali language with the purpose of making the previous system well organized and more efficient.  

Automated license plate recognition systems are available for commercial use with great cost for other languages used in mainly fields like - traffic history check, security maintenance, parking, etc. 

\section{Objectives}
The main objective of this thesis is to find out an efficient way to implement the ALPR system for Bengali digital license plates. The main objective is divided into sub objectives like - turning the system into a full working application which should be open sourced and organized for further use. 

\section{Challenges}
For fulfilling objectives there are few challenges to conquer. These challenges are -
	\begin{enumerate}
		\item reviewing and finding lacking of previous systems
		\item implement multiple systems to get a better review and future planning
        \item gather license plate image dataset
        \item preprocessing dataset properly and handling different cases
        \item implementing two different main sections - license place extraction and character recognition properly for Bengali language.
        \item comparing and performance enhancement of the developed system.
	\end{enumerate}
    
\section{Motivation}
As there is no established ALPR system in Bangladesh although few thesis related to this field already been done, it is time to make something properly usable and complete. As the main thought of making a real world effect, it is needed to find and implement a way to make a working Bengali ALPR system.

\section{Approaches}
Primarily the approach of this thesis has to be able to make a system working with the limited resources with low cost. Industrial ALPR systems are far more expensive and takes a huge amount of time, money, data which is not affordable in the scope if this thesis. So, ordinary camera images are used as dataset with minimal resolution.

The license plate extraction part is done trying few traditional techniques and also with some out of the box techniques which will be described in the later chapters. And further iterations of tuning the current system will be made in the next section of time frame for this thesis work.

Optical character recognition is the second part of this research work. This is itself a big field which includes past researches for Bengali language. Most of those researches are not of good quality to implement directly in this thesis. But, the datasets can be re-used to for training purposes. So, two approaches are under consideration- using Google's Tesseract OCR and using previous works for Bangla OCR. Both will be tried in the next part of this thesis work. 

\section{Thesis Structure}
The structure of this thesis report is mainly divided into three parts - introductory discussion including history and other background knowledges related to this field, the technical implementation and review, result analysis and conclusion.


\end{document}