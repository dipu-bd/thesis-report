\documentclass{standalone}
\usepackage{standalone}

\begin{document}

\section{Overview}
We separate our process into several stages. Each stages contains several steps in it. An overview of all stags in our process is shown in Figure \ref{fig:ProcessOverview}. The process will be described in more details in the following sections.

\begin{figure} 
	\centering
    {\small
      A flow chart will be here containing:\\
      1. Input Image\\
      2. Preprocessing\\
      3. Plate localization\\
      4. Region analysis\\
      5. Plate recognition
    }
	\caption{An overview of the plate detection procedure.} 
	\label{fig:ProcessOverview}
\end{figure}

\begin{description}
\item [Input Image] is a 24-bit colored image with red, green and blue channels.
\item [Preprocessing] stage applies several operations on input image to convert it into an image suitable for feature analysis for later stages.
\item [Plate detection] analyses content of the image to identify and extract the plate like regions.
\item [Region analysis] cleans the extracted plate (removing borders, noise etc), and segments the plate into characters. Also removes bad estimations.
\item [Plate recognition] makes the image ready for {\it Tesseract}.
\end{description}

\end{document}