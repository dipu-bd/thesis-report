\documentclass{standalone}
\usepackage{standalone}

\begin{document}
\section{Segmentation}
The goal of this stage is to separate all characters into different image to send them to OCR for recognition. We used horizontal and vertical projections in this step.


\subsection{Horizontal Segmentation}
Horizontal projection is done my taking mean of all columns across the rows of the image. Figure \ref{fig:HorizontalProjection} shows a graph of mean values across the rows. To calculate this mean we followed the formula from Equation \ref{eq:HorizontalProjection}.

\begin{equation} \label{eq:HorizontalProjection}
H = \sum^{height}_{i=1}{\dfrac{ \sum^{width}_{j=1}{ I_{ij} } }{ width } }
\end{equation}

\begin{figure}
\centering
{\it include some graphics }
\caption{Horizontal projection of a plate.}
\label{fig:HorizontalProjection}
\end{figure}

We set the cutoff line to be $\geq 1$. and separated the entire plate into two segments (Showing in Figure \ref{fig:HorizontalSegments}. 

\begin{figure}
\centering
\begin{subfigure}{0.9\textwidth}
  \centering
  %\includegraphics[width=.8\linewidth]{./img/sample/stage13.jpg}
  \caption{Upper Segment}
\end{subfigure}
\begin{subfigure}{0.9\textwidth}
  \centering
  %\includegraphics[width=.8\linewidth]{./img/sample/stage13.jpg}
  \caption{Lower Segment}
\end{subfigure}
{\it include two horizontal segments }
\caption{Horizontal segments of a plate.}
\label{fig:HorizontalSegments}
\end{figure}




\subsection{Vertical Segmentation}

\end{document}