\documentclass{standalone}
\usepackage{standalone}

\begin{document}


\subsection{Feature Extraction}
We have extracted 27 most basic features that shall help us to identify a character. We For each segmented characters we extracted same amount of features and passed it to the input of neural network for classification. Our selecting of feature effectively reduced the number of computations and made the system more efficient. We shall describe our approach to extract features here in brief.

\subsection{Converting to Normalized image}
First we convert the image into normalized binary image, which has only two values- 1 for on-character pixels, and 0 for background pixels.

\subsubsection{Feature 1 and 2}
There is a function in {\it numpy} library called {\it count\_nonzero} that counts the non-black pixels of the image. We used it to count the total non-black pixels of the initial image. This is our second feature.
\begin{lstlisting}[language=Python]
feature[1] = width * height
total_pixels = numpy.count_nonzero(img)
feature[2] = total_pixels
\end{lstlisting}

\subsubsection{Feature 3 and 4}
We divided the image into two parts horizontally (Figure \ref{fig:FeatureHor}). Our 3rd feature is the ratio between non-black pixels in the upper part and total pixels, and 4th feature considers total non-black pixels in lower part.
\begin{lstlisting}[language=Python]
center_x = width // 2
center_y = height // 2
up_pixels = numpy.count_nonzero(img[:center_x, :])
feature[3] = up_pixels / total_pixels
down_pixels = numpy.count_nonzero(img[center_x:, :])
feature[4] = down_pixels / total_pixels
\end{lstlisting}

\begin{figure}
\centering
\includegraphics{img/feature/hor}
\caption{Dividing image into upper and lower areas}
\label{fig:FeatureHor}
\end{figure}

\subsubsection{Feature 5 and 6}
Similarly we divided the image vertically (Figure \ref{fig:FeatureVer}), and extracted our features from left and right side respectively.
\begin{lstlisting}[language=Python]
left_pixels = numpy.count_nonzero(img[:, :center_y])
feature[5] = left_pixels / total_pixels
right_pixels = numpy.count_nonzero(img[:, center_y:])
feature[6] = right_pixels / total_pixels
\end{lstlisting}

\begin{figure}
\centering
\includegraphics{./img/feature/ver}
\caption{Dividing image into left and right areas}
\label{fig:FeatureVer}
\end{figure}


\subsubsection{Feature 7 to 10}
Now we divide the image into four parts (Figure \ref{fig:FeatureFours}). And calculate the non-black pixels ratio on each parts.
\begin{lstlisting}[language=Python]
up_left_pixels = numpy.count_nonzero(img[:center_x, :center_y])
feature[7] = up_left_pixels / total_pixels
up_right_pixels = numpy.count_nonzero(img[:center_x, center_y:])
feature[9] = up_right_pixels / total_pixels
bottom_left_pixels = numpy.count_nonzero(img[center_x:, :center_y])
feature[8] = bottom_left_pixels / total_pixels
bottom_right_pixels = numpy.count_nonzero(img[center_x:, center_y:])
feature[10] = bottom_right_pixels / total_pixels
\end{lstlisting}

\begin{figure}
\centering
\includegraphics{./img/feature/fours}
\caption{Dividing image into 4 areas}
\label{fig:FeatureFours}
\end{figure}

\subsubsection{Feature 11 to 26}
Next we divide the image into eight parts (Figure \ref{fig:FeatureEights}). And similarly calculate the feature for each parts.
\begin{lstlisting}[language=Python]
qx = height // 4
qy = width // 4
four = range(0, 4)
for i in four:
    for j in four:
        x1 = i * qx
        x2 = (i + 1) * qx
        y1 = j * qy
        y2 = (j + 1) * qy

        pixels = np.count_nonzero(img[x1:x2, y1:y2])
        feature[11 + 4*i + j] = 100 * pixels / total_pixels
    # end if
# end if
\end{lstlisting}

\begin{figure}
\centering
\includegraphics{./img/feature/eights}
\caption{Dividing image into 16 parts}
\label{fig:FeatureEights}
\end{figure}

\subsubsection{Feature 27}
The final feature is calculated using the average of the euclidean distance between all the black pixels from central point (Figure \ref{fig:FeatureDist}).
\begin{equation}
feature[27] = \dfrac{1}{total\_pixels} \times \sum_{y}{\sum_{x}{ \sqrt{(x-center\_x)^2 \times (y-center\_y)^2} }}
\end{equation}


\begin{figure}
\centering
\includegraphics[width=0.8\linewidth]{./img/feature/dist}
\caption{Distances of non-black pixels from center.}
\label{fig:FeatureDist}
\end{figure}


\end{document}