\documentclass{standalone}
\usepackage{standalone}

\begin{document}
\chapter*{Abstract}

Automated license plate recognition is important in many contexts like- security and law enforcement, monitoring vehicles, automated parking control etc. To enable these automated services in Bangladesh, we are reviewing and combining several established methods in this paper. The task of recognizing vehicle number plate consists of three stage: plate detection, plate extraction, and character recognition. Each stage has many sub-steps. For every sub-steps, we have reviewed many methods, and chosen the one that proved to be the best solution after thorough testing and observation. The main objective of this research is to gain high accuracy using as less CPU Time as possible, keeping into consideration the facts like- lighting conditions, vehicle motion, noisy plates, segmented words in the input image. Our primary target of this thesis is to extract a clean image of license plates of private or community vehicles. Although we target our system to be able to detect standard license plates, we also tested our methods on non-standard plates. We have gained an overall accuracy of {\bf 98.41\%} in plate detection, {\bf 82.25\%} in plate extraction, and {\bf 94.11\%} in character segmentation, resulting in {\bf 76.19\%} overall accuracy using the available dataset. In terms of efficiency, the runtime of our plate detection module is {\bf 0.841} seconds, plate extraction is {\bf 0.132} seconds, character segmentation required {\bf 0.009} seconds, and average total runtime of the entire system is {\bf 0.982} seconds; in our 64-bit machine having 4.00GB main memory and Intel(R) Core(TM) i3-4005U CPU @ $2 \times 1.70GHz$.

\vspace{1.0cm}
%\clearpage

\paragraph*{Keywords:} Digital Image Processing, Automated License Plate Recognition (ALPR), Automated Number Plate Recognition(ANPR), Bangla Number Plate, Number Plate Detection.

\end{document}
