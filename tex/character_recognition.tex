\section{Character Recignition}

For licesnse plate recognition there are few ways to complete the character recigntion part. The most common technique can be divided into two steps. At first, the characters from the previous step are provided to a recognition engine that can be coded or used as a package. For example, Open source Tesseract OCR is a great choice for character recognition engine. It generates text output from the given image of characters if it is well trained. In the second approach is to implement bespoke recognition methods. For each individual character, features are computed. For recognizing the characters, OCR system will extract features first, then compare them against patterns that have been implemented earlier (Juntanasub and Sureerattanan - 2005). Alginahi (2011) tried  to extract 88 features for each character following the same approach, then compared these features with previously extracted features. Songke and Yixian (2011) used histogram to extract characters and template matching for recognition. Later when the neural network techniques got famous, Ghosh, Sharma, Islam, Biswas, and Akter - 2011 used them for character recognition. The module is trained first and then used for recognizing characters using the previously given taining feature dataset.