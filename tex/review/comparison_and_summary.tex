\section{Comparisons}

The accuracy of license plate recognition system can be described in three parts - license plate detection, character recogntion, overall accuracy. The table below gives an overview related to the approach and problems of some previous research work-

\begin{table}[htb]
	\centering
	\label{tab:error5}
	\begin{tabular}{|l|c|c|c|}
		\hline
		\multicolumn{1}{|c|}{Author and year}
		& \multicolumn{1}{|c|}{Detection (\%)}
		& \multicolumn{1}{|c|}{Recogntion (\%)} 
		& \multicolumn{1}{|c|}{Overall (\%)} \\ 
		\hline
		    
		Kong et al. (2005)\cite{Kong2005} & 96.1 \% &  & \\ \hline
		Juntanasub and Sureerattanan (2005) \cite{juntanasub2005car} & 92 \%  &  & \\ \hline
		Waterhouse (2006)\cite{waterhouse-2006} & 96 \% & 83 and 93\% & \\ \hline
		Huang, Chen, Chang, and Sandnes (2009)\cite{hsieh2009real} & 96.7 \% & 97.1 \% & 93.9 \% \\ \hline
		Alginahi (2011)\cite{alginahi2011automatic} & 98.3 \% & 98.63 \% & 94.9 \% \\ \hline
		Maglad (2012)\cite{maglad2012vehicle} & 95 \% & 91 \% & \\ \hline
		Joarder, Mahmud, Tasnuva, Kawser, Bulbul (2012)\cite{joarder2012bangla} & 92.1 \% & 84.16 \% & 75.51 \\ \hline
	\end{tabular}
	\caption{Comparison Table}
\end{table}

\par
\section{Summary}


Review of previous works shows that the edge detection approach can be used effectively to find the license plate region. Geometrical feature analysis is an appropriate approach for this. Some researches used their own recogntion system, others used OCR engines like - teserract. Some parts of all these research can be used further in this work, others are of no use for future work.
