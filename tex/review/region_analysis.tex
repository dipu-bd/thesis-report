\section{Region Analysis}

When the feature analysis is done, it is possible that several regions of the image can be candidates for the license plate portion. For solving this problem some filtering is needed to lower the number of candidates by analyzing and removing them. A very common technique to do the task is to consider features of the bounding rectangle of candidate region. For example we can consider - aspect ratio, size width, height and location for candidate filtering. For example, calculating the rectangles features for determining the license plate probable region is a common trick (Yoon,
Lee, and Lee - 2009) \cite{yoon_lee_lee_2009}. Also, Abolghasemi and Ahmadyfard (2009) \cite{abolghasemi_ahmadyfard_2009}, Tsai, Wu,
Hsieh, and Chen (2009) \cite{Tsai2009} and Kong et al. (2005) \cite{Kong2005} used some geometrical features of the candidate
regions to detect the actual plate region. Following this with
further addition of techniques Gazcón, Chesñevar, and Castro (2012) \cite{gazcon2012automatic} proposed that if the region can be divided into well defined siz sub sections then it is the license plate region. Other researches like Hung and Hsieh (2010) \cite{hsieh2009real} used geometrical features technique
with a more or less the same technique by considering characters into
of the number plate while filtering the candidate regions. A slightly different approach for determining the license plate region was proposed by Chen, Chen, Huang, and Wang
(2011) \cite{chen2011license} by using the high density edges followed by horizontal projection.
