\section{Character Segmentation}

For recognition of the characters sometimes it is needed to segment them before sending them to the recognition system. Different research used different techniques for this part of work.

Using histogram to segment the characters is a common trick first used by Songke and Yixian (2011) and Huang et al. (2008) \cite{HUANG2008}. The technique was to use the accumulation or summation of the vertical and horizontal projections. Connected-component labeling algorithm can be used for the purpose of character segmentation which was used by Yoon et al. (2009) \cite{yoon_lee_lee_2009} and Maglad (2012) \cite{maglad2012vehicle}. The algorithm acts in two steps. Detection of the connected characters those are in black while the background is white. Then labeling those character blobs with bounding box. This technique was implemented by Liaqat (2011) \cite{liaqat2011mobile} using two algorithms, Canny for detecting the edges then contour finding algorithm to find the connected edges. Labeling and segmentation can be done using other methods. Next recognition step used many of the information gained in this step. All these information can be passed to a method for recognizing characters. Tsai et al. (2009) \cite{tsai2009recognition} tried further judgement of this step to ensure that the selected characters makes the actual license plate, not some other part of the image. The conditions to pass this test were that the width, and height of character rectangles must be similar, character density should be in a range, and the width must be less than the height.